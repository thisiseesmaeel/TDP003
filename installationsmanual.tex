\documentclass{TDP003mall}
\usepackage{graphicx}
\graphicspath{ {home/TDP003/Installationsmanual/} }


\newcommand{\version}{Version 1.0}
\author{Philip och Ismail}
\title{Installationsmanual}
\date{2020-09-16}
\rhead{Philip\\
Ismail\\}



\begin{document}
\projectpage
\section{Kommentar}
För högre betyg hade jag gärna sett lite information om vad syftet med de olika installationerna är. Vad är Flask? Varför behöver man installera Flask för att köra er portfolio? Ett avsnitt om felsökning hade också gett pluspoäng. Hur kan vi till exempel kontrollera att Flask har installerats korrekt?

\section{Revisionshistorik}
\begin{table}[!h]
\begin{tabularx}{\linewidth}{|l|X|l|}
\hline
Ver. & Revisionsbeskrivning & Datum \\\hline
1.0 & Första utkastet av installationsmanualen & 20-09-16 \\\hline
\end{tabularx}
\end{table}

\section{Emacs}
Installationsmanual för \textbf{Emacs} i Linux-system.
För att installera Emacs på ditt system, ska den följande kommandot användas i terminalen:
\begin{itemize}  
\item
 \textbf{sudo apt install emacs}
\end{itemize}
Om man vill, man kan kontrollera vilken version av \textbf{Emacs} är installerad med följande kommandot:
\begin{itemize}
\item
  \textbf{emacs --version}
\end{itemize}

\section{PyCharm}
Installationsmanual för \textbf{PyCharm} i Linux-system.
För att installera \textbf{PyCharm}, ska den följande kommandot användas i terminalen:
\begin{itemize}
\item
sudo snap install pycharm-community --classic
\end{itemize}
När installationen är klar, så ska det se ut såhär:

\includegraphics [scale=0.5] {installpycharm}

Vilket innebär att installationen är klar.

\section{Magit}
Installationsmanual för \textbf{Magit} i Linux-system. 
För att installera Magit, behövs \textbf{Melpa} också vilket ingår i dessa steg.
\begin{itemize}
\item
Skriv \textit{cd} i terminalen för att flytta till \textit{home}
\item
  Skriv \textit{emacs .emacs} för att öppna \textbf{.emacs}
\item
  Lägg till följande text:

  
\begin{verbatim}
 (require 'package)
 (add-to-list 'package-archives
         '(''melpa-stable''.''https://stable.melpa.org/packages/)t)
\end{verbatim}

  
  \includegraphics [scale=0.5] {ettan}
  
\item
  Spara filen och stäng den. Öppna upp den igen precis som i steg 2.
\item
  Skriv \textit{M-x package-refresh-contents}
\item
  Skriv \textit{M-x list-packages}
\item
  Leta igenom listan och tryck på \textit{magit}
\item
  Klicka på ''install'' och klicka ''yes'', \textbf{Magit} borde nu installeras.
  
  \includegraphics [scale=0.5] {tva}
  
\item
  För att säkerställa att den har installerats kan följande kommando skrivas:
  \textbf{M-x magit-status}

  
\end{itemize}


  Då borde ''Git repository'' dyka upp i minibuffern.

  \includegraphics [scale=0.5] {tre}


\section{Flask och Jinja2}
\textit{venv} står för ''virtual environments'' och behövs ifall datorn innehåller olika pythonprojekt och t.o.m. olika versioner av python. Dessa kan störa varandra och \textit{venv} gör så att ditt pythonprojekt isoleras från dina andra pythonprojekt.

\textit{Jinja2 ingår i \textbf{Flask}}

\begin{itemize}
\item
  Skapa ett direktiv med kommandot \textit{''mkdir Portfolio''}
\item
  Flytta dig till direktivet med kommandot \textit{''cd Portfolio''}
\item
  Skriv \textit{''python3 -m venv venv''}
  \subitem
  Om det inte funkar måste du ladda ner \textit{venv}. Skriv \textit{''sudo apt-get install python3-venv''}, logga in och skriv \textit{''y''}. När den har installerats klart, skriv kommandot i \textbf{3} igen.
\item
  Skriv \textit{''. venv/bin/activate''.} Nu kommer \textit{''(venv)''} att stå i terminalen framför ditt namn.
\item
  Inom \textit{venv}, skriv \textit{''pip install Flask''.} Nu har Flask installerats i din virtuella maskin.

  För att avsluta \textit{venv} kan du antingen skriva \textit{''exit''} eller trycka \textit{Ctrl+d}. Då kommer hela terminalen att stängas. För att öppna \textit{venv} igen kan du följa steg 4. 

  
  \includegraphics [scale=0.5] {jinja}
  
\end{itemize}

\end{document}
