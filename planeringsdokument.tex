\documentclass{TDP003mall}



\newcommand{\version}{Version 1.1}
\author{Philip och Ismail}
\title{Planeringsdokument}
\date{2020-09-08}
\rhead{Philip och Ismail}




\begin{document}
\projectpage

\tableofcontents
\newpage

\section{Revisionshistorik}
\begin{table}[!h]
\begin{tabularx}{\linewidth}{|l|X|l|}
\hline
Ver. & Revisionsbeskrivning & Datum \\\hline
1.1 & Planeringsdokument kompletterat & 15/9 2020 \\\hline
\end{tabularx}
\end{table}


\section{Introduktion}
I projektet kommer en hemsida som fungerar som en portfolio skapas där framtida projekt kan visas och laddas upp. Webbplatsen kommer att bestå av två huvuddelar: ett datalager och ett presentationslager. Presentationslagret kommer att bestå av HTML- och CSS-kod medan datalagret består av en JSON-fil. Båda dessa huvuddelar kommer att integreras med API.

\subsection{Deadlines}
\begin{table}[!h]
\begin{tabularx}{\linewidth}{|l|X|}
\hline
Vecka & Beskrivning \\\hline
37 & Planeringsdokumentet \\\hline
38 & LoFi-prototyp, grundläggande installationsmanual \\\hline
39 & Första utkast av projektplanen, första versionen av den gemensamma installationsmenualen \\\hline
40 & Konstruktiv kritik av installationsmanualen eller testerna, bug-fixar, godkänt datalager \\\hline
41 & Presentationslarget \\\hline
42 & Redovisning, portfolion publicerad, första versionen av systemdokumentationen \\\hline
43 & Testdokumentation, indviduell reflektionsdokument, fixad systemdokumentation \\\hline
\end{tabularx}
\end{table}

\newpage

\section{Planering}
\subsection*{Vecka 37}
Planeringsdokumentet ska vara klar på torsdag 10/9.

Retrospekt: Den inskickade planeringsdokumentet var inkomplett och behövdes kompletteras. Kompletteringen kommer därför att ske i vecka 38.

\subsection*{Vecka 38}
\begin{table}[!h]
\begin{tabularx}{\linewidth}{|X|l|l|l|}
\hline
Moment & Hård deadline & Milstolpe & Tid \\\hline
Lofi-prototypen & 17/9 & 15/9 & 1h \\\hline
Installationsmanualen & 17/9 & 17/9 & 8h \\\hline
Komplettering av planeringsdokumentet & N/A & 15/9 & 3h\\\hline
\end{tabularx}
\end{table}

Tisdagen dedikeras till Lofi-prototypen och kompletteringen av planeringsdokumentet, medans installationsmanualen arbetas under onsdag och torsdag.

LoFi-prototypen kommer att vara en skiss på hur hemsidan kommer att se ut och vilka funktioner den har. Skissen kommer att vara på papper. Den ska innehålla en tab för projekt och tekniker som har använts. Den ska vara tydlig.

I den grundläggande installationsmanualen ska instruktioner för hur man installerar och underhåller vår portfolio finnas med. Målgruppen är någon som vet och kan använda Linux men inte särskilt mycket.

\begin{itemize}
\item
  Tillvägagången skall visas med konkreta och förklarade exempel, kommandon och skärmdumpar.
\item	
  Systemkrav, programvarukrav och förutsättningar skall dokumenteras, och vanliga fel beskrivas.
\item	
  Installationsmanualen skall innehålla ett avsnitt om felsökning och underhåll. Beskriv hur nya projekt kan läggas till och hur gamla kan modifieras.
\item	
  Bilder/figurer ska ha förklarande figurtexter med nummer, och de ska refereras till i texten
\item	
  Det skall finnas information om författare, datum, dokumentversion, projektnamn, kurs och totalt antal sidor (dessa saker är bra att ha med i en dokumentmall, t.ex. i sidhuvud/sidfot på varje sida). 
\end{itemize}

\subsection*{Vecka 39}
\begin{table}[!h]
\begin{tabularx}{\linewidth}{|X|l|l|l|}
\hline
Moment & Hård deadline & Milstolpe & Tid \\\hline
Första utkast av projektplan & 24/9 & 24/9 & 2h \\\hline
Första versionen av den gemensamma installationsmanualen & 24/9 & 24/9 & 5h \\\hline
Datalager och API & N/A & * & *\\\hline
\end{tabularx}
\end{table}
* = Så mycket som vi hinner

Första utkastet av projektplanen och första versionen av den gemensamma installationsmanualen arbetas under tisdag och torsdag medans datalagret arbetas om de två förstnämnda momenten blir klara tidigt och på fredagen.

Planeringsdokumentet kommer att vara bas till projektplanen så mycket av planeringsdokumentet kommer att kopieras och klistras in. Sedan kommer vi också att utöka projektplanen om det behövs.

Arbetsbörden på den första versionen av den gemensamma installationsmanualen kommer att bero på allas installationsmanualer.

Datalagret och API:n kommer att påbörjas på fredag. Då kommer alla dokumentationer att läsas så som kravspecifikationen och API-instruktionerna så att vi kan påbörja det nästa vecka.

\subsection*{Vecka 40}
\begin{table}[!h]
\begin{tabularx}{\linewidth}{|X|l|l|l|}
\hline
Moment & Hård deadline & Milstolpe & Tid \\\hline
Feedback & 1/10 & 1/10 & 5h* \\\hline
Projektplan korrigerad & 1/10 & 29/9 & 6h* \\\hline
Datalagret arbetat och godkänd & 2/10 & 1/10 & 12h\\\hline
\end{tabularx}
\end{table}
* = Uppskattad extremfall

Under tisdag och torsdag kommer alla moment för denna vecka att arbetas, samt datalagret och API:n. 

Feedback syftar på antingen den gemensamma installationsmanualen eller de gemensamma testerna av installationsmanualen. Eventuella brister ska då vara korrigerade.

Projektplanen ska vara helt fri från eventuella brister denna vecka.

Datalagret kommer att arbetas. Den kommer innehålla bl.a. Flask, Jinja2 osv.

\subsection*{Vecka 41}
Tisdag och torsdag dedikeras till presentationslagret och att påbörja systemdokumentationen. Det finns inga hårda deadlines denna vecka men milstolpen är att få presentationslagret klar 8/10. Momentet uppskattas ta 16 timmar. Om det finns tid över kan systemdokumentationen börja läsas på.

\subsection*{Vecka 42}
\begin{table}[!h]
\begin{tabularx}{\linewidth}{|X|l|l|l|}
\hline
Moment & Hård deadline & Milstolpe & Tid \\\hline
Första versionen av systemdokumentationen & 15/10 & 15/10 & 12h \\\hline
Portfolio publicerad & 15/10 & 15/10 & 6h \\\hline
\end{tabularx}
\end{table}

Portfolion och den första versionen av systemdokumentationen kommer att arbetas under måndag, onsdag och torsdag.

En systemredovisning kommer att hållas torsdag 15/10 också.

\subsection*{Vecka 43}
\begin{table}[!h]
\begin{tabularx}{\linewidth}{|X|l|l|l|}
\hline
Moment & Hård deadline & Milstolpe & Tid \\\hline
Testdokumentation inlämnad & 22/10 & * & 8h \\\hline
Reflektionsdokument inlämnad & 22/10 & * & 2h \\\hline
Systemdokumentation fri från eventuella brister & 22/10 & * & 6h \\\hline
\end{tabularx}
\end{table}
* = Vet inte

I testdokumentationen kommer testning av portfolions funktioner att dokumenteras.

Reflektionsdokumentet kommer att innehålla lärdomar av att ha arbetat med projektet.

Systemdokumentationen ska vara fri från eventuella brister.

\end{document}
